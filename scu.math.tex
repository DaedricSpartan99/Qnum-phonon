%-------------------------------------------------------------
% 导言区
\documentclass[dvipsnames]{beamer}
\usepackage[utf8]{inputenc}
\usepackage{hyperref}
\usepackage[T1]{fontenc}
\usepackage{chemformula}

\usepackage{listings}

\usepackage{latexsym,multicol,booktabs}
%\usepackage[dvipsnames]{xcolor}
%\usepackage{calligra}
\usepackage{amsmath,amssymb,siunitx}
%\usepackage{BOONDOX-cal}
\usepackage{bm}		%   数学公式宏包
\usepackage{graphicx,pstricks,stackengine}      
\usepackage{subcaption}

\lstset{
  basicstyle=\fontsize{8}{13}\selectfont\ttfamily
}

%   个人信息
\author{Raffaele Ancarola}
\title{Phonon calculation in a crystal using DFPT}
\subtitle{Project 8 - CH-452}
\institute{EPFL}    
\date{February 2, 2022}

\logo{\includegraphics[height=0.3cm]{logo.png}\hspace*{0.15cm}}

%   设置主题文件    后缀名为.sty的文件是一个主题文件,初学者不要修改sty文件
\usepackage{SCU}



%   defs 特殊字体
\def\cmd#1{\texttt{\color{red}\footnotesize $\backslash$#1}}
\def\env#1{\texttt{\color{blue}\footnotesize #1}}


%   简化定理命令
\newtheorem{thm}{Theorem}[theorem]

%\newcommand{\hbar}{\mathchar'26\mkern-9mu h}
\newcommand{\diff}[2]{\frac{\partial #1}{\partial #2}}
\newcommand{\ddiffx}[1]{\frac{\partial^2 }{\partial^2 #1}}
\newcommand{\ddiff}[2]{\frac{\partial^2 #1}{\partial^2 #2}}
\newcommand{\ddiffd}[3]{\frac{\partial^2 #1}{\partial #2 \partial #3}}
\newcommand{\electrons}{\textcolor{red}{electrons}}
\newcommand{\nuclei}{\textcolor{blue}{nuclei}}
%—-------------------------------------------------------------
% 正文区

\begin{document}

	% 封面
	\begin{frame}
		\titlepage
	\end{frame}
	
	% 目录
	\begin{frame}
		\tableofcontents[sectionstyle=show,subsectionstyle=show/shaded/hide,subsubsectionstyle=show/shaded/hide]
	\end{frame}
		
	
	%—------------------------------------------------------
	% 正文
	\section{Theory}
	
	    \begin{frame}{Phonons: overview}
		
		\begin{itemize}
		\item Oscillations of atomic structure in a crystal
		\item Harmonic oscillators: $\propto e^{i(qR+\Omega(q)t)} $
		\item Dispersion relation: $\Omega(q)$
		\end{itemize}
		
		\begin{exampleblock}{Phonon types:}
		\begin{itemize}
		\item In phase: \textit{acoustic} phonons, $\Omega(q= 0) = 0$.
		\item Out of phase: \textit{optical} phonons, $\Omega(q=0) \neq 0$.
		\end{itemize}
		\end{exampleblock}
		
		\begin{exampleblock}{\textit{DOS} (density of states)}
		\[ D(\Omega) = \frac{V}{V_q} \frac{1}{v_g(\Omega),} \quad v_g(\Omega) = \diff{\Omega}{q}\]
		\end{exampleblock}
		\end{frame}
	
		\begin{frame}{System description \cite{qetuto}}
		The potential involves all coulomb interactions:
		\[ V(r, R) = V_{nn}(R) + V_{ne}(r,R) + V_{ee}(r) \]
		$r$ electron positions, $R$ nuclei positions.
		\[i \hbar \diff{\Phi}{t}(r,R;t) = \left(-\sum_I\frac{\hbar^2}{2M_I}\ddiffx{R_I} - \sum_i \frac{\hbar^2}{2m} \ddiffx{r_i} + V(r, R) \right)\Phi(r,R;t)\]
		\begin{block}{Born-Opperheim approximation (adiabatic)}
		Electrons are much faster and more lightweight than nuclei:
		\[ \Phi(r,R;t) \approx \phi(R) \psi(r|R) e^{-\frac{i\hat{E}t}{\hbar}}\]
		\end{block}
		\end{frame}
		
		\begin{frame}{System description}
		\begin{align*}
		&\left(-\sum_i\frac{\hbar^2}{2m} \ddiffx{r_i} + V(r,R) \right) \psi(r|R) = E(R) \psi(r|R) & \text{\electrons} \\
		&\left(-\sum_I\frac{\hbar^2}{2M_I} \ddiffx{R_I} + E(R) \right) \phi(R) = \hat{E} \phi(R) & \text{\nuclei} 
		\end{align*}
		We treat \nuclei ~as classical particles, so at equilibrium forces are vanishing:
		\[F_I^\alpha = - \diff{E(R)}{R_I^\alpha} = 0 \]
		\end{frame}
		
		\begin{frame}{Phonon dispersion relation}
		Expand $E(R)$ around equilibrium positions until order two:
	    \begin{equation} \label{eq:forces}
	    \sum_{J,\beta}(C_{IJ}^{\alpha\beta}(R) - M_I \omega^2 \delta_{IJ} \delta_{\alpha\beta})U_J^\beta(R) = 0 \quad C_{IJ}^{\alpha\beta} = \ddiffd{E(R)}{R_I^\alpha}{R_J^\beta}
	    \end{equation}
	    a set of independent harmonic oscillators for nuclei $I$ and coordinate $\alpha$. 
	    \begin{itemize}
	    \item $C_{IJ}^{\alpha\beta}(R)$ are called \textit{inter-atomic force constants}
	    \item $U_J^\beta(R)$ represent the oscillation amplitudes
	    \item \alert{This is a diagonalization problem!}
	    \end{itemize}
		\end{frame}
		
		\begin{frame}{Phonon dispersion relation}
		Taking fourier transform on eq. \ref{eq:forces}, get an equivalent formulation:
	    \[ \sum_{t,s}(\tilde{C}_{st}^{\alpha\beta}(q) - M_s \Omega^2(q) \delta_{st} \delta_{\alpha\beta})U_t^\beta(q) = 0\]
	    \begin{block}{Introduce a perturbation $u_s(q)$:}
	    \begin{align*}
	    &R_I[u_s](q) = R_l + \tau_s + u_s(q)e^{iqR_l} \\
	    &\tilde{C}_{st}^{\alpha\beta}(q) = \sum_R e^{-iqR}C_{st}^{\alpha\beta} = \frac{1}{N_c} \ddiffd{E}{u_s^\ast(q)}{u_t(q)} 
	    \end{align*}
	    \end{block}
	    \alert{Evaluate $\tilde{C}_{st}^{\alpha\beta}(q)$ using \texttt{DFPT}!}
		\end{frame}
		
		\begin{frame}{DFPT (density functional perturbation theory)}
		Introduce a perturbation $\delta V_0(\lambda)$ on the system potential $V(r,R)$, then energy and density can be Taylor expanded:
		\begin{align*}
		&E_\lambda = E + \lambda \diff{E}{\lambda} + \frac{1}{2} \lambda^2 \ddiff{E}{\lambda} + \dots \\
		&\varrho_\lambda = \varrho + \lambda \diff{\varrho}{\lambda} + \frac{1}{2} \lambda^2 \ddiff{\varrho}{\lambda} + \dots
		\end{align*}
		The \textcolor{orange}{linear response} can be evaluated as function of the perturbed potential.
		\end{frame}
		
		\begin{frame}{DPFT (density functional perturbation theory)}
		\begin{block}{Corollary of Hellmann-Feynman theorem}
		\[\ddiffd{E}{\lambda}{\mu} = \int dr \diff{V(r)}{\lambda} \diff{\varrho(r)}{\mu} + \varrho(r) \ddiffd{V(r)}{\lambda}{\mu}\]
		\end{block}
		\onslide<2->
		\alert{In practice:} 
		\begin{itemize}
		\item set $\lambda = u_s^\alpha(q)$, $\mu = u_t^\beta$, then find $\tilde{C}_{st}^{\alpha\beta}$.
		\item the unperturbed \electrons ' system needs to be relaxed first
		\item $\forall q$ the perturbed \electrons ' system needs to be relaxed too!
		\end{itemize}
		\end{frame}
		
		\begin{frame}{DPFT: Dielectric properties}
		Apply an external electric field to the system $\vec{E}$, then the dielectric tensor can be expressed as a linear response too:
		\[ \varepsilon_\infty^{\alpha\beta} = \delta_{\alpha\beta} + \frac{1}{N_c} \ddiffd{E}{\vec{E_\alpha}}{\vec{E_\beta}} \]
		\end{frame}
		
		\begin{frame}{Solve \electrons ' equation in a crystal \cite{castep}}
		Periodic boundary conditions in lattice $\implies$ Bloch plane wave basis:
		\[\psi_q^j(r) = \sum_G^{|G| < G_{max}} c_{q}^j(G) \cdot e^{i(q+G)r}, \quad E_{cut\psi} = \frac{\hbar^2 G_{max}^2}{2m} \]
		Express the \textit{Kohn-Sham} energy  functional:
		\[ E[\varrho, c_{q}^j(G)] = (K + E_H + E_{XC} + \delta V_0 + V_{en})[\varrho, c_{q}^j(G)] + V_{nn}(R) \]
		\textit{Constraint} for $\varrho[c_{q}^j(G)]$ + \textit{orthonormality} of $c_{q}^j(G)$.
		\par
		\onslide<2->
		\alert{$E[\varrho, c_{q}^j(G)]$ sums over all energy levels: set a \textbf{cutoff level} $E_{cut\varrho}$}.
		\par
		\alert{Preserve high freq. behaviour: \textbf{pseudopotentials}. \cite{castep}}
		\end{frame}
		
		
		
		\begin{frame}{Algorithm implemented in Quantum Espresso}
		\textit{Setup}: compute unperturbed solution (\texttt{pw.x}).
		\par
		\vspace{0.5cm}
		For a given family of reciprocal vectors $\{q\}$:
		\begin{itemize}
		\item<1-> Diagonalize perturbed \electrons ' equation (\texttt{ph.x})
		\item<1-> Compute \textit{inter-atomic force constants} in fourier space.
		\item<2-> Convert forces into real space (\texttt{q2r.x}).
		\item<3-> Diagonalize forces tensor (eq. \ref{eq:forces}) and get dispersion relation as eigenvalue (\texttt{matdyn.x}).
		\end{itemize}
		
		\end{frame}
		
		
		
	\section{Molecule bulk \ch{MoS2}}
	
	    \begin{frame}{Crystal structure: overview  \cite{osti_1202268}}
	        \begin{figure}
	        \centering
	        \begin{subfigure}{0.65\textwidth}
	        \includegraphics[scale=0.3]{MoS2_horizonthal.png}
	        \caption{Horizontal view.}
	        \end{subfigure}
	        \begin{subfigure}{0.3\textwidth}
	        \includegraphics[width=\textwidth]{MoS2_vertical.png}
	        \caption{Vertical view.}
	        \end{subfigure}
	        \caption{Crystal structure visualization of bulk \ch{MoS2} restricted to its primitive cell. The primitive cell is \textit{trigonal} shaped. \cite{vesta}}
	        \end{figure}
	    \end{frame}
	    
	    \begin{frame}{Crystal structure: parameters \cite{osti_1202268}}
	    \begin{columns}[T] % align columns
			\begin{column}{.40\textwidth}
			\begin{block}{Primitive cell}
    		\begin{itemize}
    		\item $a = b = 3.190 \; \si{\angstrom}$
    		\item $c = 14.879 \;\si{\angstrom}$
    		\item $\alpha = \beta = 90 \si{\degree}$
    		\item $\gamma = 120 \si{\degree}$
    		\end{itemize}
    		\end{block}
			\end{column}%
			\hfill%
			\begin{column}{.65\textwidth}
			\begin{block}{Atoms relative positions}
    		\begin{tabular}{c|c|c|c}
    	     & a & b & c \\ \hline
    		\ch{Mo} & 0.3333 & 0.6667 & 0.25 \\ \hline
    		\ch{Mo} & 0.6667 & 0.3333 & 0.75 \\ \hline
    		\ch{S} & 0.3333 & 0.6667 & 0.8552 \\ \hline
    		\ch{S} & 0.3333 & 0.6667 & 0.6448 \\ \hline
    		\ch{S} & 0.6667 & 0.3333 & 0.3552 \\ \hline
    		\ch{S} & 0.6667 & 0.3333 & 0.1448
    		\end{tabular}
    		\end{block}
			\end{column}%
		\end{columns}
	    \end{frame}
	    
	    \begin{frame}{Reciprocal space and phonon calculation path \cite{SeeK}} \label{fr:path}
	    \begin{columns}[T] % align columns
			\begin{column}{.40\textwidth}
			\begin{figure}
			\centering
			\includegraphics[width=\textwidth]{brillouin.png}
			\caption{I$^{st}$ brillouin zone of bulk \ch{MoS2} (hexagonal). $\Gamma$ is defined as the center.}
			\end{figure}
			\end{column}
			\hfill
			\begin{column}{.60\textwidth}
			\begin{block}{High symmetry frequency path}\label{eq:path}
			    $\Gamma - M - K - \Gamma - A - L - H - A$
			\end{block}
			\begin{figure}
			\centering
			\includegraphics[width=0.6\textwidth]{brillouin_path.png}
			\caption{Path corresponding to the path described above.}
			\end{figure}
			\end{column}
		\end{columns}
	    \end{frame}
	
	%-------------------------------------------------------	
	\section{\textit{Quantum espresso} setup and simulation}
	
	    \begin{frame}{Workflow}
	    \begin{columns}[T] % align columns
	    \begin{column}{.70\textwidth}
	    \includegraphics[scale=0.5]{steps.pdf}
	    \end{column}
	    \begin{column}{.30\textwidth}
	    Scripts:
	    \begin{itemize}
	    \item \texttt{pw.x}
	    \item \texttt{ph.x}
	    \item \texttt{q2r.x}
	    \item \texttt{matdyn.x}
	    \item \texttt{plotband.x}
	    \end{itemize}
	    \end{column}
	    \end{columns}
	    \end{frame}
	    
		\begin{frame}{Self consistent simulation: system and electron}	
		\begin{columns}[T] % align columns
			\begin{column}{.50\textwidth}
			\lstinputlisting{mos2-bulk.scf_system.in}
			\begin{block}{Electron occupation}
			Allow filling fraction of electrons 
			\end{block}
			\end{column}
			\hfill
			\begin{column}{.50\textwidth}
			\lstinputlisting{mos2-bulk.scf_electrons.in}
			\begin{block}{Energy cutoff parameters}
			\begin{itemize}
			\item \texttt{ecutrho}: $E_{cut\varrho}$ [\si{Ry}].
			\item \texttt{ecutwfc}: $E_{cut\phi}$ [\si{Ry}].
			\end{itemize}
			\end{block}
			\end{column}
		\end{columns}
		\end{frame}
		
		\begin{frame}{Self consistent simulation: lattice parameters}
		\lstinputlisting[basicstyle=\fontsize{7}{9}\selectfont\ttfamily]{mos2-bulk.scf_eletrocn.in}
		\end{frame}
		
		\begin{frame}{Analysis}	
		
		\begin{columns}[T] % align columns
			\begin{column}{.50\textwidth}
			\texttt{\color{red}{mos2-bulk.phX.in}}
			\lstinputlisting[basicstyle=\fontsize{9}{10}\selectfont\ttfamily]{mos2-bulk.phX.in}
			\begin{block}{$\Gamma \neq 0$}
			\begin{itemize}
			\item \texttt{ldisp=.false.}
			\item \texttt{epsil=.true.}
			\item Shift \texttt{K\_POINTS: 1 1 1}
			\end{itemize}
			\end{block}
			\end{column}
			\hfill
			\begin{column}{.50\textwidth}
			\texttt{\color{red}{matdyn.in}}
			\lstinputlisting[basicstyle=\fontsize{9}{10}\selectfont\ttfamily]{matdyn.in}
			\texttt{\color{red}{q2r.in}}
			\lstinputlisting[basicstyle=\fontsize{9}{10}\selectfont\ttfamily]{q2r.in}
			\end{column}
		\end{columns}
		\end{frame}
	
	%-----------------------------------------------------
	\section{Results}
	
	\begin{frame}{\texttt{scf} calculations: convergence tests on total energy}
	\begin{figure}
	\begin{subfigure}{\textwidth}
	\resizebox{\textwidth}{0.35\textheight}{
		\input{graphs/conv-cutrho.pgf}
	}
	\caption{Energy cutoff in $U(\rho)$ [\si{Ry}].}
	\end{subfigure}
	
	\begin{subfigure}{\textwidth}
	\resizebox{\textwidth}{0.35\textheight}{
		\input{graphs/conv-cutwfc.pgf}
	}
	\caption{Energy cutoff in $|\phi_n\rangle$ [\si{Ry}].}
	\end{subfigure}
	\end{figure}
	\end{frame}
	
	\begin{frame}{\texttt{scf} calculations: convergence tests on total energy}
	\begin{figure}
	\resizebox{\textwidth}{0.35\textheight}{
		\input{graphs/conv-k.pgf}
	}
	\caption{Number of \texttt{K\_POINTS} discretization along $(k_x, k_y)$.}
	\end{figure}
	\begin{columns}[T]
	\begin{column}{0.5\textwidth}
	\begin{block}{Advised values by QeTools}
	\begin{itemize}
	\item \texttt{ecutrho}: 280
	\item \texttt{ecutwfc}: 35
	\item \texttt{K\_POINTS}: 12
	\end{itemize}
	\end{block}
	\end{column}
	\begin{column}{0.5\textwidth}
	\begin{exampleblock}{Reference energy}
	Increase all values at once:
	$E_{ref} = -370.7519$ \si{Ry}
	\end{exampleblock}
	\end{column}
	\end{columns}
	\end{frame}
	
	\begin{frame}{Dispersion relation and \texttt{DOS}}
	\begin{figure}
	\begin{subfigure}{0.7\textwidth}
	\vspace{-0.7cm}
	\includegraphics[trim=0 3cm 0 1cm,angle=-90,origin=c,width=0.9\textwidth, height=0.8\textheight]{graphs/mos2.dispersions.ps}
	\end{subfigure}
	\begin{subfigure}{0.25\textwidth}
	\scalebox{1}[-1]{
	\includegraphics[trim=0 4cm 0 -2cm,clip,angle=180,origin=c,width=0.9\textwidth, height=0.75\textheight]{graphs/mos2.phdos.ps}
	}
	\end{subfigure}
	\caption{Dispersion relation $\Omega(Q)$ computed along the high symmetry path (frame \ref{fr:path}). On the right the density of states (DOS).}
	\end{figure}
	\end{frame}
	
	\begin{frame}{Dispersion relation and \texttt{DOS}: comparison with references}
	\begin{figure}
	\includegraphics[height=0.7\textheight]{graphs/phonon_dispersion.png}
	\caption{Dispersion relation detected with X-ray Rahman scattering \cite{Hans}. Each color corresponds to a direction.}
	\end{figure}
	\end{frame}
	
	\begin{frame}{Dispersion relation and \texttt{DOS}: comparison with references}
	\begin{figure}
	\includegraphics[height=0.4\textheight]{graphs/bulk_dos.png}
	\caption{Dispersion relation determined by Rahman spectroscopy and a simulated curve \cite{Molina}. Yellow points are experimental data.}
	\end{figure}
	% TODO: conclusions
	\end{frame}
	
	\begin{frame}{Dielectric tensor}
	\begin{columns}[T]
	\begin{column}{0.5\textwidth}
	\begin{equation}
	\varepsilon = 
        \begin{pmatrix}
        12.75 & 0 & 0 \\
        0 & 12.75 & 0 \\
        0 & 0 & 3.29
        \end{pmatrix} %TODO: units[\si{eV/\angstrom}]
	\end{equation}
	Error on $\varepsilon$ estimation: $0.01$
	Absolute error with reference (fig. \ref{fig:ref_eps}): $0.35$
	% TODO: refractive index?
	\end{column}
	\hfill
	\begin{column}{0.5\textwidth}
	\begin{figure}
	\includegraphics[width=\textwidth]{graphs/dielectric_reference.png}
	\caption{Reference dielectric tensor given by \textit{materials project} \cite{osti_1202268}}
	\label{fig:ref_eps}
	\end{figure}
	\end{column}
	\end{columns}
	\end{frame}
	
	%----------------------------------------------
	\section{Bibliography}
		
		\begin{frame}[allowframebreaks]
			
			\tiny\bibliographystyle{apalike}
			\bibliography{ref}	
			
			 %\tiny\bibliographystyle{alpha}
            
		\end{frame}

	%-------------------------------------------
	\begin{frame}
		\begin{center}
			{\Huge \emph {\textrm{Thank  ~you!}}}
		\end{center}
	\end{frame}

\end{document}